%===============================================================================
% LaTeX sjabloon voor de bachelorproef toegepaste informatica aan HOGENT
% Meer info op https://github.com/HoGentTIN/latex-hogent-report
%===============================================================================

\documentclass[dutch,dit,thesis]{hogentreport}

% TODO:
% - If necessary, replace the option `dit`' with your own department!
%   Valid entries are dbo, dbt, dgz, dit, dlo, dog, dsa, soa
% - If you write your thesis in English (remark: only possible after getting
%   explicit approval!), remove the option "dutch," or replace with "english".

\usepackage{lipsum} % For blind text, can be removed after adding actual content

%% Pictures to include in the text can be put in the graphics/ folder
\graphicspath{{graphics/}}

%% For source code highlighting, requires pygments to be installed
%% Compile with the -shell-escape flag!
\usepackage[section]{minted}
\usemintedstyle{solarized-light}
\definecolor{bg}{RGB}{253,246,227} %% Set the background color of the codeframe

%% Change this line to edit the line numbering style:
\renewcommand{\theFancyVerbLine}{\ttfamily\scriptsize\arabic{FancyVerbLine}}

%% Macro definition to load external java source files with \javacode{filename}:
\newmintedfile[javacode]{java}{
    bgcolor=bg,
    fontfamily=tt,
    linenos=true,
    numberblanklines=true,
    numbersep=5pt,
    gobble=0,
    framesep=2mm,
    funcnamehighlighting=true,
    tabsize=4,
    obeytabs=false,
    breaklines=true,
    mathescape=false
    samepage=false,
    showspaces=false,
    showtabs =false,
    texcl=false,
}

% Other packages not already included can be imported here

%%---------- Document metadata -------------------------------------------------
% TODO: Replace this with your own information
\author{Arne Moens}
\supervisor{Dhr. M. Asselberg }
\cosupervisor{Dhr. S. Van Hoof}
\title[Optionele ondertitel]%
    {Het opstellen van een interactief, \\ dynamisch en
        gepersonaliseerd dashboard \\met behulp van widgets voor het opvolgen van GPS tracking systemen:\\
        onderzoek en proof of concept}
\academicyear{\advance\year by -1 \the\year--\advance\year by 1 \the\year}
\examperiod{1}
\degreesought{\IfLanguageName{dutch}{Professionele bachelor in de toegepaste informatica}{Bachelor of applied computer science}}
\partialthesis{false} %% To display 'in partial fulfilment'
%\institution{Internshipcompany BVBA.}

%% Add global exceptions to the hyphenation here
\hyphenation{back-slash}

%% The bibliography (style and settings are  found in hogentthesis.cls)
\addbibresource{bachproef.bib}            %% Bibliography file
\addbibresource{../voorstel/voorstel.bib} %% Bibliography research proposal
\defbibheading{bibempty}{}

%% Prevent empty pages for right-handed chapter starts in twoside mode
\renewcommand{\cleardoublepage}{\clearpage}

\renewcommand{\arraystretch}{1.2}

%% Content starts here.
\begin{document}

%---------- Front matter -------------------------------------------------------

\frontmatter

\hypersetup{pageanchor=false} %% Disable page numbering references
%% Render a Dutch outer title page if the main language is English
\IfLanguageName{english}{%
    %% If necessary, information can be changed here
    \degreesought{Professionele Bachelor toegepaste informatica}%
    \begin{otherlanguage}{dutch}%
       \maketitle%
    \end{otherlanguage}%
}{}

%% Generates title page content
\maketitle
\hypersetup{pageanchor=true}

\input{voorwoord}
\input{samenvatting}

%---------- Inhoud, lijst figuren, ... -----------------------------------------

\tableofcontents

% In a list of figures, the complete caption will be included. To prevent this,
% ALWAYS add a short description in the caption!
%
%  \caption[short description]{elaborate description}
%
% If you do, only the short description will be used in the list of figures

\listoffigures

% If you included tables and/or source code listings, uncomment the appropriate
% lines.
%\listoftables
%\listoflistings

% Als je een lijst van afkortingen of termen wil toevoegen, dan hoort die
% hier thuis. Gebruik bijvoorbeeld de ``glossaries'' package.
% https://www.overleaf.com/learn/latex/Glossaries

%---------- Kern ---------------------------------------------------------------

\mainmatter{}

% De eerste hoofdstukken van een bachelorproef zijn meestal een inleiding op
% het onderwerp, literatuurstudie en verantwoording methodologie.
% Aarzel niet om een meer beschrijvende titel aan deze hoofdstukken te geven of
% om bijvoorbeeld de inleiding en/of stand van zaken over meerdere hoofdstukken
% te verspreiden!

%%=============================================================================
%% Inleiding
%%=============================================================================

\chapter{\IfLanguageName{dutch}{Inleiding}{Introduction}}%
\label{ch:inleiding}

Doordat we steeds meer in de mogelijkheid zijn om grote hoeveelheden gegevens te verzamelen, wordt de noodzaak voor data analyse steeds groter. Dashboards zijn één van de meest gebruikte hulpmiddelen om een visuele voorstelling te geven van de belangrijkste informatie waardoor snel  correcte beslissingen kunnen genomen worden of waardoor je gemakkelijk de laatste gegevens kan raadplegen. 
Sensolus is een technologiebedrijf dat zich bezighoudt met het ontwikkelen en aanbieden van IoT-oplossingen voor bedrijven. Het biedt track- en trace-producten aan waarmee klanten in realtime hun goederen en bedrijfsmiddelen kunnen volgen en beheren. Door gebruik te maken van geavanceerde cloud-software, draadloze communicatieprotocollen en sensoren, kan Sensolus nauwkeurige locatiegegevens leveren die bedrijven helpen hun efficiëntie te verhogen, kosten te besparen en meer controle over hun bedrijfsprocessen te krijgen. Dankzij GPS-technologie en IoT maakt Sensolus het mogelijk om real-time locatiegegevens te verzamelen en klanten waardevolle informatie te bieden voor een betere optimalisatie van hun bedrijfsvoering.  

\section{\IfLanguageName{dutch}{Probleemstelling}{Problem Statement}}%
\label{sec:probleemstelling}

Het bedrijf Sensolus bied op haar web platform momenteel enkel voorgedefinieerde dashboards aan. Omdat niet elke gebruiker alle informatie nodig heeft die beschikbaar is op deze voorgedefinieerde dashboards en de informatie verspreid staat op verschillende dashboards, zou het handig zijn mocht de gebruiker de mogelijkheid hebben om een gepersonaliseerde dashboard samen te stellen. Het bedrijf wil dus een praktische en complete oplossing om haar klanten het gebruik van hun platform makkelijker en overzichtelijker te maken.

\section{\IfLanguageName{dutch}{Onderzoeksvraag}{Research question}}%
\label{sec:onderzoeksvraag}

Wat is de beste manier om een dashboard volledig dynamisch te maken, is dit met behulp van reeds bestaande tools of een geheel eigen programma? 
Welke tools bestaan er al voor het maken van dashboards en wat zijn hun voor- en nadelen?


\section{\IfLanguageName{dutch}{Onderzoeksdoelstelling}{Research objective}}%
\label{sec:onderzoeksdoelstelling}

Het doel van deze bachelorproef is om te onderzoeken welke manier van werken de beste is voor dit bedrijf. Dit zal onderzocht worden aan de hand van de literatuurstudie en aan de hand van een proof-of-concept die enkele methodes zal uittesten. Uit het onderzoek moet blijken welke manier van werken de beste is voor het bedrijf zodat zij er verder mee aan de slag kunnen gaan.
\section{\IfLanguageName{dutch}{Opzet van deze bachelorproef}{Structure of this bachelor thesis}}%
\label{sec:opzet-bachelorproef}

% Het is gebruikelijk aan het einde van de inleiding een overzicht te
% geven van de opbouw van de rest van de tekst. Deze sectie bevat al een aanzet
% die je kan aanvullen/aanpassen in functie van je eigen tekst.
De rest van deze bachelorproef is als volgt opgebouwd:
In hoofdstuk 2 ~\ref{ch:stand-van-zaken} zal de huidige stand van zaken worden toegelicht. Hier gaat het over wat een dashboard juist is hoe het is ontstaan. Ook gaat het hier over de huidige mogelijke tools en hoe deze gebruikt kunnen worden. Ze worden hier initieel vergeleken om dan in het volgende hoofdstuk verder mee aan de slag te gaan.
In hoofdstuk 3 “Methodologie” zal het onderzoek zelf uitgevoerd worden aan de hand van een proof-of-concept waar er verschillende tools zullen gebruikt worden om hetzelfde vooraf gedefinieerde resultaat te kunnen bekomen. 
In hoofdstuk 4 “Conclusie” zal er worden terug gekeken worden op de literatuurstudie en het onderzoek zelf. Hieruit zullen we dan besluiten welke methode het beste is, zodat we deze kunnen aanraden aan het bedrijf.

\chapter{\IfLanguageName{dutch}{Stand van zaken}{State of the art}}%
\label{ch:stand-van-zaken}

% Tip: Begin elk hoofdstuk met een paragraaf inleiding die beschrijft hoe
% dit hoofdstuk past binnen het geheel van de bachelorproef. Geef in het
% bijzonder aan wat de link is met het vorige en volgende hoofdstuk.

% Pas na deze inleidende paragraaf komt de eerste sectiehoofding.

Binnen verscheidene organisaties wordt de term “dashboard” gebruikt om een systeem te beschrijven dat data visualiseert die handig is voor het maken van beslissingen. Een dashboard dient de gebruiker te informeren zonder ze af te leiden van het eigenlijke doel. Daarom gebruikt men op dashboards veelal verschillende grafieken, tabellen, kaarten, icoontjes, … . Om de gebruiker zo een samenvatting te geven van de verzamelde data, waarop die zich dan kan baseren om conclusies te maken. Het juiste design van een dashboard is zeer belangrijk. De meeste dashboards zijn gemaakt om zo veel mogelijk data in 1 keer weer te geven. Dit terwijl ze eigenlijk enkel de belangrijke data op de juiste zouden moeten weergeven. (Janes e.a., 2013)

\section{\IfLanguageName{dutch}{Inhoud dashboard bepalen}{Define dashboard content}}%
\label{sec:inhoudDashboardBepalen}
Hoe je de inhoud van je dashboard bepaald is zeer belangrijk. Voor men begint aan het maken van een dashboard is het belangrijk om een analyse te maken. Hier wordt onderscheid gemaakt tussen de belangrijke en minder belangrijke zaken. Maar de manier waarop iets afgebeeld wordt is ook zeer belangrijk. Deze keuzes kunnen gemaakt worden aan de hand van het GQM-model of Goal Question Meausrement model. De Goal staat voor het doel, wat willen we juist bereiken. Welke zaken willen we te weten komen aan de hand van dit dashboard. Question laat ons zoeken in onze data naar welke informatie ons helpt ons doel te bereiken. Welke informatie draagt bij aan het dashboard om ons doel te bereiken. Eens we deze twee zaken weten kunnen we over gaan naar Measurement. Hier wordt bepaald hoe we onze data juist moeten afbeelden. Er zijn veel verschillende manieren om data af te beelden. We selecteren hieruit de beste voor ieder deeltje van onze data. Zo bekomen we op het einde van onze analyse een goed dashboard dan aan de hand van duidelijk weergegeven informatie voldoet aan onze vragen en duidelijke antwoorden schept.( Janes e.a., 2013)

\section{\IfLanguageName{dutch}{Huidige staat dashboards}{Current state dashboards}}%
\label{sec:huidigeStaatDashboards}
Er is reeds onderzoek gedaan naar het ontwerpen van custom en dynamic dashboards. Omdat het maken van dashboards relatief makkelijker en goedkoper is dan vroeger, is het maken van gepersonaliseerde dashboards een vraagstuk naar waar we kunnen kijken. Voorlopig zijn er vooral statische dashboards te vinden op de meeste websites. Indien ze toch custom zijn, zijn ze vooral ontworpen voor een bepaalde rol. Zo worden er op voorhand verschillende custom dashboards gemaakt gebaseerd op verschillende rollen, dit aangezien een CEO niet hetzelfde wil weten als een manager van een bepaald deel binnen het bedrijf of de gewone gebruiker van buiten het bedrijf. Dynamische dashboards worden nog maar
zeer zelden toegepast in de praktijk. Dynamische dashboards zullen hun layout aanpassen aan de hand van data verkregen tijdens run-time. Het verzamelen van die data gebeurt door te kijken met welke delen van het dashboard de user het meest in aanraking komt. Hieruit kan via analyse geconcludeerd worden welke informatie het belangrijkst is voor de gebruiker. Dan kan op het dashboard weergegeven worden wat de gebruiker het vaakst bekijk en gebruikt. De beste oplossing is om de twee te combineren en zowel statisch al dynamisch te werken en dus op een hybride manier. (Kruglov e.a., 2021)

Er zijn drie aanpassingsniveaus van dynamic dashboards. Er is het eerste level, hierin kunnen gebruikers zelf enkele eenvoudige parameters meegeven. Aan de hand van die parameters zal de gebruiker dus gefilterde resultaten terug krijgen. Ook kan hij hier kiezen welke componenten hij wil zien op het dashboard. Dan is er het tweede level, hier kunnen gebruikers zelf berekeningen uitvoeren op de data en kiezen hoe deze juist afgebeeld wordt. Ten slotte is er nog een
derde level, de gebruikers kunnen op dit level de data zelf aanpassen.(Ji e.a., 2014)

\section{\IfLanguageName{dutch}{Wat is de stand van zaken binnen het bedrijf}{What is the state of the business}}%
\label{sec:huidigeStaatBedrijf}
Het bedrijf Sensolus bevindt zich momenteel op het eerste aanpassingsniveau. Ze hebben zelf widgets gemaakt waar de klant parameters aan kan meegeven om zo gerichte data te zien. Maar veel van hun dashboards bevatten nog een statische structuur. Veel van hun dashboards zijn niet aanpasbaar en bevatten geen optie om de paarameters aan te passen. Deze dashboards zijn gemaakt met behulp van geïntegreerde	BI (Business Intelligence) tools. Voorlopig gebruiken ze 3 verschillende BI tools: QuickSight, PowerBI and Cumul.IO. Ze gebruiken deze tools om de klant een duidelijker beeld te scheppen naar de klanten toe hoe het met hun product gaat.  Deze bi tools zijn zeer handig om data weer te geven. Maar je kan er naast het afbeelden van vooraf bepaalde data zeer weinig mee doen. De mogelijkheden om deze data dan te bewerken of te filteren is zeer beperkt tot zelf onmogelijk.

\subsection{\IfLanguageName{dutch}{QuickSight}{QuickSight}}%
\label{sec:quickSight}
QuickSight is een BI tool gemaakt door Amazon. De tool helpt je om je data beter te begrijpen. Met QuickSight kan je makkelijk verschillende componenten gebruiken in je eigen apps. Zo maak je je eigen dashboards met behulp van de embedded visualisatie componenten. Op de app zelf kan je ook gebruik maken van QuickSight Q, dit is een machine learning (ML) gestuurder Q&A tool. Aan deze tool kan je in natuurlijke tekst vragen stellen en kom krijg je meteen een duidelijk antwoord terug. Zo kunnen gebruikers interactief data opvragen zonder enkel naar een dashboard te kijken. Ook deze QuickSight Q zoekbalk zou je in je eigen applicatie kunnen steken. Maar hiervoor heb je wel de Enterprise Edition nodig van het programma. Met QuickSight kan je aan de hand van de free-from layouts zelf je eigen dashboards samen stellen en deze, je kiest zo zelf wat waar staat. Ook kan je aan de hand van parameters zelf bepalen hoe je welke data exact te zien krijgt. Het is ook mogelijk om eigen componenten in de  dashboards te maken, zoals foto's videos, web paginas of zelf externe applicaties. Het is zeer eenvoudig om QuickSite in je eigen website te gebruiken via de 1-click embedding functionaliteit. Deze funcutionaliteit vind je in wikis, SharePoint, Google sites en nog veel meer, maar het is ook eenvoudig om deze te gebruiken in volledig zelf geschreven websites.
(https://aws.amazon.com/blogs/big-data/amazon-quicksight-2021-in-review/)

\subsection{\IfLanguageName{dutch}{Power BI}{Power BI}}%
\label{sec:powerBI}
Power BI is a Buisness Intellegence en visualisatie tool gemaakt door Microsoft. Power BI is een cloud-gebaseerde verzameling van apps, softwate en connectors die ruwe bedrijfsgegevens verzamelt en opslaat. Op basis van die ruwe bedrijfsgegevens zal de tool dan verschillende samenstellingen maken en de data op verschillende manieren interpreteren om ze zo op een duiderlijk en interactief dashboard weer te geven. Wat zijn de beste punten van Power BI? Aangezien het een tool is gemaakt door Microsoft werkt het ook goed samen met andere Microsoft producten. Vooral dan met Exel. Het is zeer eenvoudig om grote hoeveelheden data te raadplegen in 1 keer, wel 100 miljoen rijen tegenover de 1 miljoen van Exel. Je kan alles makkelijk aanpassen met behulp van R en Python. Power BI laat gebruikers toe om met meerderen tegelijk te werken aan dezelfde data. Alles gebeurt dus in real time dankzij de grote hoeveelheid resources waarover Microsoft beschikt. Power BI is ook zeer flexibel als het aankomt op het vlak van data bronnen. Je kan allerlei verschillende soorten databronnen gebruiken, zowel relationele databases als Big Data en nog vele anderen waaronder ook cloud-gebaseerde bronnen waar het bedrijf Sensolus mee werkt. En nog veel meer voordelen. Power BI maakt duidelijke rapporten mogelijk door gebruik te maken van machine learning. Deze worden op voorhand gemaakt voor de gebruiker, maar zijn ook zelf volledig aanpasbaar.
(https://www.datacamp.com/blog/all-about-power-bi)

Power BI beschikt ook over een Embedded API die de gebruiker de mogelijkheid geeft om Power BI te integreren in zijn eigen website of applicatie. Deze API is deel van het Permium Power BI packet. Deze functie laat de bedrijven toe gerichte rapporten op basis van de doelgroep weer te geven op een eigen platform. Het bied het bedrijf een extra mogelijkheid om hun klanten een duidelijk beeld van hun data weer te geven zonder daar zelf al te veel moeite in te steken. Het is dus een duidelijke en veilige manier om zonder veel extra werk een belangerijke waarde toe te voegen aan je platform.
(https://skypointcloud.com/blog/power-bi-embedded-examples/)

\subsection{\IfLanguageName{dutch}{Cumul.IO}{Cumul.IO}}%
\label{sec:cumul.IO}
Cumul.io is een belgische start-up die zich specializeert het maken van embedded Buisness Intellegence. Ze maken low-code oplossingen voor bedrijven die hun data willen analyseren binnen hun eigen platform. Ze hebben geen eigen dashboard maar laten de gebruikers toe om componenten te implementeren in hun eigen platform. Het is een snel groeiend platform dat begon in Leuven maar nu ook al kantoren heeft in Genk en New York. Hun product zorgt voor een eenvoudige integratie van analitishce raporten in een eigen systeem door middel van het plug en play principe. 
(https://www.eu-startups.com/2023/01/leuven-based-cumul-io-bags-e10-million-to-drive-better-business-decisions/)
Cumul.io werkt met wat zij noemen integrations. Dit is een structuur van dashboards die samen gebruikt moeten worden in bijvoorbeeld dezelfde applicatie. Al deze dashboards zijn embedden en worden dus met de eigen applicatie verweven. Met de integrations kan je dus ook bepalen wie wat mag zien en wat niet. Ieder dashboard heeft zijn eigen doel en alle dashboards uit één integration vormen samen een duidelijk persoonlijk geheel dat verschilt van klant tot klant.
(https://css-tricks.com/embedded-analytics-made-simple-with-cumul-io-integrations/) 
Vergelijking
Al deze BI tools die ze nu al gebruiken zijn zeer goed in wat ze doen en zijn eenvoudig te gebruiken. Er hangt wel steeds een stevig prijskaartje aan en het bedrijf kan deze wel gebruiken om per klant een uniek dashboard te maken, maar de klanten van Sensolus kunnen zelf niet kiezen wat ze juist zien en waar dat dan te zien is.
Om deze kosten te verminderen en de klanten van Sensolus meer keuze te geven naar wat ze juist zelf willen zien en wat niet kijken we nu naar enkele mogelijke alternatieven in de open-source wereld.

\section{\IfLanguageName{dutch}{Mogelijke open source embedded BI alternatieven }{Possible open source embedded BI alternatives }}%
\label{sec:opensourceAlternatieven}
\subsection{\IfLanguageName{dutch}{Superset}{Superset}}%
\label{sec:superset}
Superset is een BI tool gemaakt door de Apache Software Foundation. Het is een moderne en volledig open source data verkenning en visualisatie platform. Superset wordt reeds gebruikt door vele grote bedrijven zoals Airbnb, Lyft, Twitter, Udemy en Dropbox. Superset laat gebruikers toe om zelf dashboard samen te stellen doormiddel van drag en drop. Aangezien Superset een no-code web-based interface heeft dient de gebruiker geen voorkennis van SQL te hebben. Het omvormen van data naar duidelijke gestructureerde grafieken gebeurt allemaal automatisch en is volledig door de gebruiker zelf aan te passen. Het is ook zeer eenvoudig om Superset in je eigen website of applicatie te integreren met behulp van de Embedded SDK. Het gebruiken van Superset brengt ook een zeer uitgebreide keuze aan verschillende soorten grafieken met zich  mee.
(https://dropbox.tech/application/why-we-chose-apache-superset-as-our-data-exploration-platform)

\subsection{\IfLanguageName{dutch}{Redash}{Redash}}%
\label{sec:redash}
Redash is een zeer wijd verspreide BI tool. Het heeft een web-gebaseerde BI tool, maar bied ook een open-source versie aan die je zelf op een privé server kan laten draaien. Deze BI tool maakt het makkelijk om verschillende data warehouses te integreren, je kan ook snel SQL querries schrijven en hiermee data ophalen voor de visulatisatie. Redash zorgt er ook voor dat het eenvoudig is om een dashboard te delen met externen. Maar als die hier zelf zaken aan willen veranderen dienen zij wel een Redash account te hebben. De gebruikers dienen voor hun specifieke data op te vragen wel SQL te kennen, maar worden hierbij sterk geholpen door het programma door de Schema Browser en autocomplete functionaliteit. Alle grafieken kunnen ook worden geüpdatet op zelf te kiezen tijdstippen en intervallen.
(https://hevodata.com/learn/redash/) 

\subsection{\IfLanguageName{dutch}{Metabase}{Metabase}}%
\label{sec:metabase}
Metabase is een open-source BI tool. Het staat de gebruiker toe om te kijken naar de database en hier analytische rapporten van te maken om, die kunnen dienen om belangrijke beslissingen te maken. Metabase herlaad automatisch ieder uur alle grafieken zodat ze up to date zijn met de laatste data. Je kan ook zelf kiezen wanneer deze data herladen wordt aan de hand van een vooraf bepaald tijdstip of een bepaald interval. Metabase helpt ook bij het schrijven van makkelijke tot zeer complexe queries. Er is een lage instap drempel voor niet technisch aangelegde gebruikers, deze gebruikers kunnen gebruik maken van de duidelijke interface. Data is ook makkelijk te exporteren naar CSV, XLSX en andere formaten. Het maken van eigen dashboards met verschillende grafieken is hierdoor eenvoudig. Het is ook weer mogelijk om deze BI tool te integreren in je eigen web applicatie. Maar er zijn minder mogelijke grafieken.
(https://aniekan.blog/2022/03/07/business-intelligence-with-metabase/)


\section{\IfLanguageName{dutch}{Vergelijking}{Comparison}}%
\label{sec:vergelijking}
Al deze open-source BI tools zijn kosteloos naast de kost om ze effectief ergens op een server te laten draaien. Ze zijn allemaal implementeerbaar in eigen applicaties en kunnen aangepast worden naar behoren van de gebruiker. Door het feit dat men bij Redash altijd een account nodig heeft en bij Metabase er minder visualisatie opties zijn, lijkt het het best om verder te gaan met Superset.


\input{methodologie}

% Voeg hier je eigen hoofdstukken toe die de ``corpus'' van je bachelorproef
% vormen. De structuur en titels hangen af van je eigen onderzoek. Je kan bv.
% elke fase in je onderzoek in een apart hoofdstuk bespreken.

%\input{...}
%\input{...}
%...

\input{conclusie}

%---------- Bijlagen -----------------------------------------------------------

\appendix

\chapter{Onderzoeksvoorstel}

Het onderwerp van deze bachelorproef is gebaseerd op een onderzoeksvoorstel dat vooraf werd beoordeeld door de promotor. Dat voorstel is opgenomen in deze bijlage.

%% TODO: 
%\section*{Samenvatting}

% Kopieer en plak hier de samenvatting (abstract) van je onderzoeksvoorstel.

% Verwijzing naar het bestand met de inhoud van het onderzoeksvoorstel
\input{../voorstel/voorstel-inhoud}

%%---------- Andere bijlagen --------------------------------------------------
% TODO: Voeg hier eventuele andere bijlagen toe. Bv. als je deze BP voor de
% tweede keer indient, een overzicht van de verbeteringen t.o.v. het origineel.
%\input{...}

%%---------- Backmatter, referentielijst ---------------------------------------

\backmatter{}

\setlength\bibitemsep{2pt} %% Add Some space between the bibliograpy entries
\printbibliography[heading=bibintoc]

\end{document}
