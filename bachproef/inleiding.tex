%%=============================================================================
%% Inleiding
%%=============================================================================

\chapter{\IfLanguageName{dutch}{Inleiding}{Introduction}}%
\label{ch:inleiding}

Doordat we steeds meer in de mogelijkheid zijn om grote hoeveelheden gegevens te verzamelen, wordt de noodzaak voor data analyse steeds groter. Dashboards zijn één van de meest gebruikte hulpmiddelen om een visuele voorstelling te geven van de belangrijkste informatie waardoor snel  correcte beslissingen kunnen genomen worden of waardoor je gemakkelijk de laatste gegevens kan raadplegen. 
Sensolus is een technologiebedrijf dat zich bezighoudt met het ontwikkelen en aanbieden van IoT-oplossingen voor bedrijven. Het biedt track- en trace-producten aan waarmee klanten in realtime hun goederen en bedrijfsmiddelen kunnen volgen en beheren. Door gebruik te maken van geavanceerde cloud-software, draadloze communicatieprotocollen en sensoren, kan Sensolus nauwkeurige locatiegegevens leveren die bedrijven helpen hun efficiëntie te verhogen, kosten te besparen en meer controle over hun bedrijfsprocessen te krijgen. Dankzij GPS-technologie en IoT maakt Sensolus het mogelijk om real-time locatiegegevens te verzamelen en klanten waardevolle informatie te bieden voor een betere optimalisatie van hun bedrijfsvoering.  

\section{\IfLanguageName{dutch}{Probleemstelling}{Problem Statement}}%
\label{sec:probleemstelling}

Het bedrijf Sensolus bied op haar web platform momenteel enkel voorgedefinieerde dashboards aan. Omdat niet elke gebruiker alle informatie nodig heeft die beschikbaar is op deze voorgedefinieerde dashboards en de informatie verspreid staat op verschillende dashboards, zou het handig zijn mocht de gebruiker de mogelijkheid hebben om een gepersonaliseerde dashboard samen te stellen. Het bedrijf wil dus een praktische en complete oplossing om haar klanten het gebruik van hun platform makkelijker en overzichtelijker te maken.

\section{\IfLanguageName{dutch}{Onderzoeksvraag}{Research question}}%
\label{sec:onderzoeksvraag}

Wat is de beste manier om een dashboard volledig dynamisch te maken, is dit met behulp van reeds bestaande tools of een geheel eigen programma? 
Welke tools bestaan er al voor het maken van dashboards en wat zijn hun voor- en nadelen?


\section{\IfLanguageName{dutch}{Onderzoeksdoelstelling}{Research objective}}%
\label{sec:onderzoeksdoelstelling}

Het doel van deze bachelorproef is om te onderzoeken welke manier van werken de beste is voor dit bedrijf. Dit zal onderzocht worden aan de hand van de literatuurstudie en aan de hand van een proof-of-concept die enkele methodes zal uittesten. Uit het onderzoek moet blijken welke manier van werken de beste is voor het bedrijf zodat zij er verder mee aan de slag kunnen gaan.
\section{\IfLanguageName{dutch}{Opzet van deze bachelorproef}{Structure of this bachelor thesis}}%
\label{sec:opzet-bachelorproef}

% Het is gebruikelijk aan het einde van de inleiding een overzicht te
% geven van de opbouw van de rest van de tekst. Deze sectie bevat al een aanzet
% die je kan aanvullen/aanpassen in functie van je eigen tekst.
De rest van deze bachelorproef is als volgt opgebouwd:
In hoofdstuk 2 ~\ref{ch:stand-van-zaken} zal de huidige stand van zaken worden toegelicht. Hier gaat het over wat een dashboard juist is hoe het is ontstaan. Ook gaat het hier over de huidige mogelijke tools en hoe deze gebruikt kunnen worden. Ze worden hier initieel vergeleken om dan in het volgende hoofdstuk verder mee aan de slag te gaan.
In hoofdstuk 3 “Methodologie” zal het onderzoek zelf uitgevoerd worden aan de hand van een proof-of-concept waar er verschillende tools zullen gebruikt worden om hetzelfde vooraf gedefinieerde resultaat te kunnen bekomen. 
In hoofdstuk 4 “Conclusie” zal er worden terug gekeken worden op de literatuurstudie en het onderzoek zelf. Hieruit zullen we dan besluiten welke methode het beste is, zodat we deze kunnen aanraden aan het bedrijf.
